\documentclass{beamer}
\usepackage{beamerthemesplit}
\usepackage{graphicx}
\usepackage{listings}

\title{dnode}
\subtitle{remote execution in the kingdom of node.js}
\author{James Halliday (substack)}
\date{\today}

\begin{document}

\frame{\titlepage}

\frame{
    \frametitle{call remote functions!}
    {
        remote method invocation
    }
}

\frame{
    \frametitle{At a glance}
    {
        dnode is 
        \begin{itemize}
        \item{fundamentally asynchronous}
        \item[]{
            \begin{itemize}
            \item no return values
            \end{itemize}
        }
        \pause
        \item{symmetric}
        \item[]{
            \begin{itemize}
            \item both sides can call the other's methods
            \end{itemize}
        }
        \pause
        \item{abstract}
        \item[]{
            \begin{itemize}
            \item use it on the browser or the server
            \end{itemize}
        }
        \end{itemize}
    }
}

\frame{
    \frametitle{Zing example server!}
    {
        \lstinputlisting{code/zing/server.js}
    }
}

\frame{
    \frametitle{Zing example client!}
    {
        \lstinputlisting{code/zing/client.js}
        \pause
        \lstinputlisting{code/zing/console}
    }
}


\frame{
    \frametitle{How it doesn't work}
    {
        \begin{itemize}
        \item{Function.prototype.toString()}
        \item{eval()}
        \item{$<$script$>$ tag injection}
        \end{itemize}
        
        \pause
        NOT THESE!
        
    }
}

\frame{
    \frametitle{How it does work}
    {
        When a function gets called...
        
        \begin{itemize}
        \item{a recursive walk pulls out all the functions from the arguments}
        \item{function paths and IDs sent alongside}
        \item{handles cycles too}
        \end{itemize}
        
        \pause
        \small
        \lstinputlisting{code/protocol_message.json}
        \normalsize
        
        \pause
        This transformation is recursive!
    }
}

\frame{
    \frametitle{It's callbacks all the way down!}
    {
        \lstinputlisting{code/turtles.js}
    }
}

\frame{
    \frametitle{OOP is just hashes of functions}
    {
        \lstinputlisting{code/objects.js}
    }
}

\frame{
    \frametitle{Good news for you!}
    {
        No need to write your own
        \begin{itemize}
        \item{method dispatcher}
        \item{state machine}
        \item{serialization protocol}
        \end{itemize}
    }
}

\frame{
    \frametitle{Retrofitting the server}
    {
        \lstinputlisting{code/browser/server.js}
    }
}

\frame{
    \frametitle{Browser code}
    {
        \small
        \lstinputlisting{code/browser/index.html}
    }
}

\frame{
    \frametitle{Git it!}
    {
        \begin{center}
        \huge
        github.com/substack/dnode
        
        \vspace{2em}
        
        github.com/substack/dnode-slides
        \end{center}
    }
}

\end{document}
